\documentclass[conference]{IEEEtran}
\IEEEoverridecommandlockouts
% The preceding line is only needed to identify funding in the first footnote. If that is unneeded, please comment it out.
\usepackage{cite}
\usepackage{amsmath,amssymb,amsfonts}
\usepackage{algorithmic}
\usepackage{graphicx}
\usepackage{textcomp}
\usepackage{xcolor}
 \usepackage{multirow}

%---------------------------------
%By Ti
%%% Кодировки и шрифты %%%
\usepackage{cmap}						% Улучшенный поиск русских слов в полученном pdf-файле
\usepackage[utf8]{inputenc}				% Кодировка utf8
\usepackage[english, russian]{babel}    % Языки: русский, английский
\usepackage[T2A]{fontenc}				% Поддержка русских букв
% \usepackage{pscyr}
\usepackage{booktabs}						% Красивые русские шрифты
%---------------------------------

\def\BibTeX{{\rm B\kern-.05em{\sc i\kern-.025em b}\kern-.08em
    T\kern-.1667em\lower.7ex\hbox{E}\kern-.125emX}}
\begin{document}

\title{Исследование эллиптической геометрической модели сердца для компьютерной многоканальной электроимпедансной кардиографии\\
%{\footnotesize \textsuperscript{*}Note: Sub-titles are not captured in Xplore and
%should not be used}
\thanks{The reported study was funded by RFBR according to the research project No 18-29-02042.}
}

\author{\IEEEauthorblockN{1\textsuperscript{st}  Alexey N. Tikhomirov}
\IEEEauthorblockA{\textit{Bauman Moscow State} \\
\textit{Technical University}\\
Moscow, Russia \\
tikhomirov.an@bmstu.ru}
\and
\IEEEauthorblockN{2\textsuperscript{nd} Andrey Briko}
\IEEEauthorblockA{\textit{Bauman Moscow State} \\
\textit{Technical University}\\
Moscow, Russia \\
briko@bmstu.ru}
\and
\IEEEauthorblockN{3\textsuperscript{rd} Nikolay Seleznev}
\IEEEauthorblockA{\textit{Bauman Moscow State} \\
\textit{Technical University}\\
Moscow, Russia \\
seleznev.nv@bk.ru}
\and
\IEEEauthorblockN{4\textsuperscript{th} Given Name Surname}
\IEEEauthorblockA{\textit{dept. name of organization (of Aff.)} \\
\textit{name of organization (of Aff.)}\\
City, Country \\
email address or ORCID}
\and
\IEEEauthorblockN{5\textsuperscript{th} Given Name Surname}
\IEEEauthorblockA{\textit{dept. name of organization (of Aff.)} \\
\textit{name of organization (of Aff.)}\\
City, Country \\
email address or ORCID}
\and
\IEEEauthorblockN{6\textsuperscript{th} Given Name Surname}
\IEEEauthorblockA{\textit{dept. name of organization (of Aff.)} \\
\textit{name of organization (of Aff.)}\\
City, Country \\
email address or ORCID}
}

\maketitle

\begin{abstract}
    абстракт

\end{abstract}

\begin{IEEEkeywords}
component, formatting, style, styling, insert
\end{IEEEkeywords}

\section{Introduction}

Stroke volume (SV), cardiac output (CO), and ejection fraction (EF) are crucial
parameters for CVS assessment. These parameters highlight the circulatory
dynamics of the heart.

Circulatory parameters in clinical research and practice are assessed with CT,
MRI, and ultrasound. These methods provide loads of valuable diagnostic
information on the heart. However, continuous monitoring is not possible with
these methods for financial and dosimetric reasons. 

Thermal dilution through pulmonary artery catheterization (PAC) is a golden
standard of SV determination. However, a set of non- or minimally invasive
procedures extensively [Kobe, Mishra 2019].

LiDCO uses the minimally invasive method: lithium chloride dilution [Linton,
Band 1993]. Despite its minimal invasiveness, this method has a drawback: the
system should be calibrated every 8 hours or if the hemodynamic condition has
changed. Also, one of the counter-indications of LiCl dilution is intolerance to
lithium. PiCCO and FloTrac use contour analysis of the pressure curve, which is
accessed invasively via the catheter. Valve regurgitation, severe arrhythmia,
and rapid changes in body temperature may affect the accuracy of measurements.

Electrical impedance methods of the CVS study are non-invasive and inexpensive.
Among them are transthoracic methods, electrical impedance tomography, and
precardial impedance cardiac plethysmography. 

Transthoracic electrical impedance plethysmography (TEIP) traces back to the
middle of the XX century and is now used with minor changes in such apparatus as
CardioScreen (Medis) and Rheo-Spectre (Neurosoft). Non-invasive methods have
fewer complications to use, but not without drawbacks. The TEIP method is used
to assess changes and trends in assessing blood parameters and not determine the
absolute values. 

Impedance cardiography (ICG) applies to both the lungs and the heart studies.
The main disadvantage is low spatial resolution in terms of assessing heart
hemodynamics problem. 

In 2002, a precardial mapping technique was proposed to expand on electrical
impedance methods for heart studies.

\section{Materials and methods}

\subsection{Precardial Impedance Cardiography Methods}

The precardial impedance cardiography combines several methods. Firstly,
precardial radial mapping method. The electrode systems are located along the
ventricle projection border onto the chest surface, perpendicular to the border.
Secondly, longitudinal-transverse mapping, in which one electrode system is
located on the chest's surface above the heart along the heart's anatomical
axis, and the other is perpendicular to the axis. 

Precardial impedance cardiography is based on the inverse problem of the
electrical impedance measurements. The geometric model is built based on the a
priori anatomical data, generally provided via the CT or MRI. 
%рассказать, что можно использовать не свежие данные
The heart is modeled with a sphere, ellipsoid, or more complex geometry. The
heart's contraction is represented by changing the parameters of the model, such
as the radius of the sphere. The experimentally recorded changes in electrical
impedance during the heart contraction are converted into changes in the
geometric model parameters based on the solution of the inverse problem, and
then the volumetric characteristics are estimated. 

The complexity of the chest and heart's geometric model forms a larger number of
parameters that describe the contraction of the heart in more detail. However,
this requires a larger number of electrode systems to obtain data for solving
the inverse problem of electrical impedance measurement. 

Hence, a more complex model allows one to obtain more information and requires
more electrode systems. A simpler model requires fewer electrode systems but is
limited in capabilities. It is necessary to find a compromise between the
model's complexity and the output information of the model. 

In the monitoring of circulatory characteristics, the vanity and simplicity of
the technique are often more critical. This paper compares two geometric models
of a homogeneous half-space with a sphere and ellipsoid inclusions. Both models
are considered for the method of precordial longitudinal-transverse cardiac
mapping. 

\subsection{Ellipsoid and Sphere models}

%\begin{figure}[tbph]
%    \centering
%    \includegraphics[width=\linewidth]{fig/1_4}
%    \caption{}
%    \label{fig:14}
%\end{figure}

In the simulation, the heart's blood is often represented as the sphere in
electrical impedance measurement problems,
%[наши статьи]
as well as in precordial mapping methods and electrical impedance tomography of
the heart. 
%[статьи по томографии]

This model is quite simple. According to the tomography data, the sphere's
parameters are determined - the radius and coordinates of the center. The
heart's contraction corresponds to a change in the radius of the sphere and
center's offset.

Using the ellipsoid model allows for a better approximation of blood in the
heart before the onset of ventricular systole. 
% статьи по эллипсу

However, the number of model parameters increases - three semiaxes of the
ellipsoid, coordinates of the center, and rotation of the ellipsoid in space
compared to one radius and the center's coordinates. The heart's contraction
corresponds to a change in the semiaxes of the ellipsoid and a shift in its
center. Also, the ellipsoid can be spatially rotated during contraction. 

The models' geometric parameters were obtained from the data of a CT of a
healthy volunteer. The study of multispiral computed tomography was carried out
in Pirogov Moscow City Clinical Hospital №1  with  Toshiba Aquilion PRIME 160
and under the patronage of employees of both the clinic and the medical center
of the BMSTU medical center. 

The study was carried out with the introduction of an iodine-containing contrast
agent Optiray 350 mg, 90 ml for the study on inspiration and 90 ml for the study
on exhalation, 50 ml at a rate of 3.5 ml/sec, and 40 ml at a rate of 3 ml/sec. 

The study was carried out on free expiration. As a result of these studies'
reconstruction, sets of slices with a time resolution of 20 series per cardio
cycle were obtained, which gives about 35-50 ms between the series. The distance
between the axial slices was 2.5 mm. The sets of sections were used to
reconstruct a 3D model of blood in the heart. 

\subsection{Sphere and Ellipsoid Approximation}
The parameters of the sphere and ellipsoid were obtained by approximating 3D models of the heart's blood.

The sphere was approximated by the criterion of the least square deviation of the sphere boundaries and real 3D geometry.

Approximation criterion \ref{eq:sphere_criteria},
$ p_i$ - a finite set of points on the surface of a 3D heart model,
$dist(p_i,Z(f))$ -  distance from the point $ p_i $ to the surface of the sphere $ Z (f) $. 
\begin{equation}
    \sum_{i=1}^{q}dist(p_i,Z(f))^2 \rightarrow min
    \label{eq:sphere_criteria}
\end{equation}

The ellipsoid was obtained according to the criterion described in . 
%todo добавить ссылку на статью BOOK
The method is peculiar because authors consider the distance from a point to the
ellipsoid $ dist (p_i, Z (f)) $ as follows: the difference between 1) the
distance from the center of the ellipsoid to the point and 2) the distance from
the center of the ellipsoid to a point on its boundary. The said point is lying
on the ray outgoing their center and passing through a given point. This
assumption's error is estimated depending on the ratio of the ellipsoid
semiaxes, and an adjustment is made. The final optimization criterion is also
the least square method (2). 

Analysis of the results and visualization of the approximation showed that the
septa of the heart, such as the interventricular septum, 
%todo ссылка на рисунок модели серца
the atrioventricular septum (Fig.~\ref{fig: wall}) forms points that distort
the approximation results, causing an increase in the average deviation of the
surface of the approximating ellipse or sphere from the outer boundary of the 3D
blood model. 
%----------------------------------------------------------------------------
\begin{figure}[tbph]
    \centering
    \includegraphics[width=0.8\linewidth]{fig/wall}
    \caption{3D model of blood in the heart (interventricular septum highlighted in blue)}
    \label{fig:wall}
\end{figure}
To eliminate this problem, it was decided to fill in the original 3D models of
the heart partitions and internal voids. For this, the 3D model was laid out on
2D images obtained by sectioning by planes parallel to the X and Y axes (Fig.~\ref{fig:algo}). 
%--------------------------------------------------------
\begin{figure}[tbph]
    \centering
    \includegraphics[width=\linewidth]{fig/algo}
    \caption{Algorithm for filling the septa of the heart in the 3D model of the heart's blood 
    (on the left - the original 3D model, on the right - the 3D model after filling the septa)}
    \label{fig:algo}
\end{figure}
%---------------------------------------------------------
Then a tangent path traversal was made of the 2D image contour with a circle radius of
20 mm (red circle), as a result of a set of obtained sections filled with septum
was going to the 3D image of the heart (Fig.~\ref{fig:algo2}).

\begin{figure}[tbph]
    \centering
    \includegraphics[width=\linewidth]{fig/algo2}
    \caption{Tangent path traversal of a 2D image with a circleto fill heart's semptum
    in the 3D models of heart's blood (on the left is white contour beforethe septum filling,
        on the right side - after the filling)}
    \label{fig:algo2}
\end{figure}

\section{Моделирование}
\section{Параметры электроимпедансного моделирования}
Для анализа полученных геометрических моделей проводилось электроимпедансное моделирование в среде конечно-элементного анализа Comsol Multiphisics.
Расположение электродных систем при моделировании соответствовало расположению электродных систем при продольно-поперечном картировании. Расстояние между токовыми электродами варьировалось от 80 до 240 мм, а отношение расстояния между токовыми электродами к расстоянию между потенциальными составляло 2 к 1.

Значения удельных сопротивлений включения и однородного полупространства представлены в таблице \ref{tab:table}.

\begin{table}[htbp]
    \caption{RESISTIVITY VALUES AT 100 KHZ USED IN THE MODEL STUDY}
    \begin{center}
        \begin{tabular}{|l|c|c|}
            \hline
            \multirow{2}{*}{\textbf{Tissue}}              &     \textbf{Resistivity values,}      &    \textbf{Resistivity value in}  \\
            &    \textbf{ Omh*m}     &   \textbf{ simulation, Omh*m }\\
            \hline
            Blood (Hct=50)           & 1.35 [21]           & 1.35      \\
            \hline
            Myocardium               & 4.6 [20]       & \multirow{7}{*}{4.2}\\
            \cline{1-2}
            \multirow{2}{*}{Muscles} & 2.7 [20]          &     \\
            \cline{2-2}
            & 1.5 – 25 [19]           &   \\
            \cline{1-2}
            Lung (deflated)          & 3.68 [20]     &         \\
            \cline{1-2}
            Lung                     & 1.6 – 10 [1]       &    \\
            \cline{1-2}
            Human thorax & \multirow{2}{*}{4.63 [19]}  &   \\
            (average)   & &   \\
            \hline
        \end{tabular}
        \label{tab:table}
    \end{center}
\end{table}
%todo поправить список литературы в таблице

Усреднение удельных сопротивлений легочной, мышечной ткани и миокарда возможно, так как рассматриваются измерения в фазу спокойного выдоха, а удельное сопротивление легочной ткани на выдохе приближается к удельному сопротивлению мышечной ткани.

\subsection{Сравние 3D моели со сферой и эллипсоидом}

По нескольким параметрам сравнивались сами модели и результаты моделирования реальной 3D геометрии крови в сердце, сферы и эллипсоида.
Во-первых, производилось сравнение изменения объема крови в сердце, то есть объема реальной 3D модели, с объемами аппроксимирующих сферы и эллипса. Оценка объемов производилась для каждого из 20 моментов времени, на которые разбивался кардиоцикл.
Во-вторых, рассматривалось движение объема крови в сердце в целом, то есть движение центра масс реальной 3D модели и аппроксимирующих её фигур.
В- третьих, сравнивались зависимости изменения импеданса в ходе сердечного цикла для различных расположений и размеров электродных систем.

\begin{figure}[htbp]
%    \centerline{\includegraphics{fig/fig1.png}}
    \centering{\includegraphics[width=\linewidth]{fig/1_4}}
    \caption{Значение R x y z ЦМ в ходе сердечного цикла}
    \label{fig:rxyz}
\end{figure}

\section{Results}

Зависимости изменения объема и координат центра масс исходной 3D модели и аппроксимирующей сферы и эллипсоида представлены на Рис.~\ref{fig:rxyz},
где на временной шкале сердечный цикл от R-зубца до R-зубца.
% по оси абсцисс

Результаты моделирования изменения импеданса в ходе сердечного цикла представлены на Рис.~\ref{real},Рис.~\ref{fig:sphere},Рис.~\ref{fig:ellipse}.
Для каждой модели представлены зависимости для электродной системы расположенной вдоль анатомической оси сердца и перпендикулярно ей.
На этих графиках так же рассматривается сердечный цикл от R-зубца до R-зубца.

\begin{figure}[tbph]
    \centering
    \includegraphics[width=\linewidth]{fig/real}
    \caption{Зависимость изменения импеданса в ходе сердечного цикла для исходной 3D модели (расположение электродной системы вдоль оси сердца - слева,перпендикулярно оси сердца - справа)}
    \label{real}
\end{figure}

\begin{figure}[tbph]
    \centering
    \includegraphics[width=\linewidth]{fig/sphere}
    \caption{Зависимость изменения импеданса в ходе сердечного цикла для сферической модели (расположение электродной системы вдоль оси сердца - слева,перпендикулярно оси сердца - справа)}
    \label{fig:sphere}
\end{figure}

\begin{figure}[tbph]
    \centering
    \includegraphics[width=\linewidth]{ellipse}
    \caption{Зависимость изменения импеданса в ходе сердечного цикла для сферической модели (расположение электродной системы вдоль оси сердца - слева,перпендикулярно оси сердца - справа)}
    \label{fig:ellipse}
\end{figure}



\section{Discussion}



\section{Conclusion}

%An excellent style manual for science writers is \cite{b7}.

\begin{thebibliography}{00}
\bibitem{b1} G. Eason, B. Noble, and I. N. Sneddon, ``On certain integrals of Lipschitz-Hankel type involving products of Bessel functions,'' Phil. Trans. Roy. Soc. London, vol. A247, pp. 529--551, April 1955.
\bibitem{b2} J. Clerk Maxwell, A Treatise on Electricity and Magnetism, 3rd ed., vol. 2. Oxford: Clarendon, 1892, pp.68--73.
\bibitem{b3} I. S. Jacobs and C. P. Bean, ``Fine particles, thin films and exchange anisotropy,'' in Magnetism, vol. III, G. T. Rado and H. Suhl, Eds. New York: Academic, 1963, pp. 271--350.
\bibitem{b4} K. Elissa, ``Title of paper if known,'' unpublished.
\bibitem{b5} R. Nicole, ``Title of paper with only first word capitalized,'' J. Name Stand. Abbrev., in press.
\bibitem{b6} Y. Yorozu, M. Hirano, K. Oka, and Y. Tagawa, ``Electron spectroscopy studies on magneto-optical media and plastic substrate interface,'' IEEE Transl. J. Magn. Japan, vol. 2, pp. 740--741, August 1987 [Digests 9th Annual Conf. Magnetics Japan, p. 301, 1982].
\bibitem{b7} M. Young, The Technical Writer's Handbook. Mill Valley, CA: University Science, 1989.
\end{thebibliography}
\vspace{12pt}
%\color{red}
%IEEE conference templates contain guidance text for composing and formatting conference papers. Please ensure that all template text is removed from your conference paper prior to submission to the conference. Failure to remove the template text from your paper may result in your paper not being published.

\end{document}
